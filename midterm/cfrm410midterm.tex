\documentclass{article}
\linespread{1.3}
\usepackage[margin=50pt]{geometry}
\usepackage{amsmath, amsthm, amssymb, amsthm, tikz, fancyhdr}
\pagestyle{fancy}
\renewcommand{\headrulewidth}{0pt}
\newcommand{\changefont}{\fontsize{15}{15}\selectfont}

\fancypagestyle{firstpageheader}
{
  \fancyhead[R]{\changefont Michael Huang \\ Midterm \\ Adekoya}
}

\begin{document}

\thispagestyle{firstpageheader}

\section*{1.}
{\Large 

\subsection*{(a)}
Of the 12 swimmers, 2 are from the UK, 2 from China, 4 from Russia , 3 from the US, and 1 from France.
Based on their rank in the competition, 
each swimmer is assigned a score from 1 to 12. So, no two swimmers can have the same score. The scoring
takes account of the countries the swimmers represent, it does not take account their individual identities.
How many dierent outcomes are possible from the point of view of the scores? Also, how many of the
possible outcomes have one US swimmer in the top three and two in the bottom three?



\subsection*{(b)}
How many ways are there to place 18 identical bowties in 5 distinguishable baskets such that
each basket contains at least 2 bowties? There is no restriction on the maximum number of bowties each
basket can have.



\subsection*{(c)}
A Greek sorority name can have either two or three letters, not necessarily distinct. How many
sorority names are possible? 
To find the number of sorority names possible with not necessarily distinct letters, we need to first find the number of sorority names possible with two letters and three letters separately. \\
To do this, 

Also, how many sorority names contain three distinct letters in alphabetical
order? The Greek alphabet has 24 letters.

}

\section*{2.}
{\Large

\subsection*{(a)}
Let A and B be two arbitrary events. Prove that $P(A^C \cap B^C) \geq 1 - P(A) - P(B)$. State any
axioms or basic results about probability that you use. (Hint: use inclusion-exclusion principle.)

\subsection*{(b)}
Let A B and C be arbitrary events. Prove that \\ 
$P(A \cup B \cup C) = P(A) + P(B) + P(C) - P(A^C \cap B \cap C) - P(A \cap B^C \cap C) - P(A \cap B \cap C^C) - 2P(A \cap B \cap C)$.\\ 
State any axioms or basic results about probability that you use. 
(Hint: This formula may also be of help to you $P(E) = P(E \cap F) + P(E \cap F^C)$.)


\subsection*{(c)}
A fair dice is thrown twice. Consider the following events:
$ E =$ "the first throw gives a 3 or a 4", and
$ F =$ "the sum of the two throws is at least 7".
Are the events $"E"$ and $"F"$ dependent or independent? Explain your answer.

}

\section*{3.}
{\Large 
The expected number of typographical errors on any given page of a book published
by ABC house is 0.3. 
When answering the following questions,be sure to explain what probability distribution you are using and why.

\subsection*{(a)}
We aim to find the probability that there will be at least one typo on a random page.


\subsection*{(b)}
Assume Sarah was tired, so she skimmed through the page. Despite not being as thorough as
she usually is, she finds a mistake. What is the probability that she will find exactly 3 more mistakes on
that page when she gathers enough strength to read it carefully?


\subsection*{(c)}
Two typists were each responsible for typing up half of Sarah's book. Sarah does not know who
typed which half. The average number of errors per page is 2.5 when typed by the first person, and 4.6
when typed by the second person. What is the probability that a randomly chosen page will have no errors?


}

\section*{4.}
{\Large 
You play a game in which you place a bet and then a fair coin is flipped. If the result is heads, you win the
amount of the bet. Otherwise, you lose the amount of your bet.
For example, if you bet \$5, then you either gain \$5 (if the result is head) or lose \$5 (if the result is tail).
You initially have \$512 = $2^9$, and decide to play successive games as follows. You bet \$1 in your first game, and stop playing if you win, earning \$1. If you lose the first game, you pay \$1 and play a second game where you place a bet of \$2. You stop playing if you win the second game, earning \$1=2-1 in the two games that you played. Otherwise, if you lose the second game, you pay another \$2 and play a third game by doubling your bet to \$4. You continue playing in this way. Each time you win, you stop. If you lose, and if you have money left (even \$1) after paying the loss, then you play another game where you double your previous bet.
You stop once you have lost all of your money. 

Let $X$ be the net winnings, that is, the amount you win in the last game 
(a negative number if you lose the final game) minus the losses for the previous games.

\subsection*{(a)}
What are the possible values of $X$? Calculate the pmf of $X$. Explain your answers.


\subsection*{(b)}
Calculate $P(X > 0)$:


\subsection*{(c)}
Calculate E($X$):


}

\end{document}
