\documentclass{article}
\linespread{1.3}
\usepackage[margin=50pt]{geometry}
\usepackage{amsmath, amsthm, amssymb, amsthm, tikz, fancyhdr}
\pagestyle{fancy}
\renewcommand{\headrulewidth}{0pt}
\newcommand{\changefont}{\fontsize{15}{15}\selectfont}

\fancypagestyle{firstpageheader}
{
  \fancyhead[R]{\changefont Michael Huang \\ Homework 7 \\ Adekoya}
}

\begin{document}

\thispagestyle{firstpageheader}

\section*{1.}
{\Large 

\subsection*{(a)}
% $X \sim \text{Exp}(\lambda)$, where $f(x) = \lambda e^{-\lambda x} \text{ for } x \geq 0$, and 0 otherwise.
We use the exponential distribution to model the waiting time for $t = 1$ hour as demonstrated in class with $\lambda = 1$, i.e., $X \sim \text{Exp}(1)$. We can therefore take P$(X > 1) = e^{-\lambda \cdot 1} = e^{-1} = $ \framebox[1.1\width]{\textbf{$\frac{1}{e}$}}

\subsection*{(b)}
We aim to find the conditional probability that a repair takes at least 10 hours, or P$(X \geq 10)$, given that its duration exceeds 9 hours, or P$(X > 9)$. This is essentially P$(X \geq 10 \mid X > 9)$. 
Using the memorylessness property of the exponential distribution to simplify $P(X \geq 10 \mid X > 9) = $ P$(X > 1) = = e^{-\lambda \cdot 1} = e^{-1} = $ \framebox[1.1\width]{\textbf{$\frac{1}{e}$}}, as we previously found.

}

\section*{2.}
{\Large

David is deciding to buy either a new car or a used car that has been driven 10,000 miles. For
each car, find the probability that he can drive it for an extra 20,000 miles in the following
scenarios.

\subsection*{(a)}
We model the total mileage that both cars can be driven before they completely break down as an
exponential random variable with mean 20 thousand miles. Since we know that E$(X)$ for an exponential variable $= \frac{1}{\lambda}$, we know that we can model the total mileage as $X ~ \text{Exp}(\frac{1}{20,000})$. \\ \\ 
The new car has been driven 0 miles, so we simply aim to find P$(X > 0 + 20,000) = \text{P}(X > 20,000) = e^{-\lambda \cdot a} = e^{-\frac{1}{20,000} \cdot 20,000} = e^{-1} = $ \framebox[1.1\width]{\textbf{$\frac{1}{e}$}} \\
The used car has been driven 10,000 miles, so we simply aim to find P$(X > 10,000 + 20,000) = e^{-\lambda \cdot a} = e^{-\frac{1}{20,000} \cdot 30,000} = e^{-\frac{3}{2}} = $ \framebox[1.1\width]{\textbf{$\frac{1}{\sqrt{e^3}}$}} \\ 

\subsection*{(b)}
We model the total mileage that both cars can be driven before they completely break down as a uniform distribution over (0, 40,000), i.e., $X \sim \text{U}(0, 40,000)$. We can therefore say that $F(x) = \text{P}(X \leq x) = \frac{x}{40,000} \text{ for } 0 < x < 40,000$. \\ \\
The new car has been driven for 0 miles, so we simply aim to find P$(X > 0 + 20,000) = 1 - \text{P}(X \leq 20,000) = 1 - \frac{20,000}{40,000} = $ \framebox[1.1\width]{\textbf{$\frac{1}{2}$}} \\ 
The used car has been driven for 0 miles, so we simply aim to find P$(X > 10,000 + 20,000) = 1 - \text{P}(X \leq 30,000) = 1 - \frac{30,000}{40,000} = $ \framebox[1.1\width]{\textbf{$\frac{1}{4}$}}

}

\section*{3.}
{\Large 

Let  $X, Y$ be continuous r.v.s with the following joint probability density function:

\[
f_{X,Y}\left(x,y\right)=\begin{cases}
e^{-x}\cdot\frac{k}{\sqrt{x}}& \,0< x,\,\, 0\le|y| \le \sqrt{x};\\
0 & \,\mbox{otherwise,}
\end{cases}
\]

\subsection*{(a)}
We know that by definition, $\int_{-\infty}^{\infty} \int_{-\infty}^{\infty} f_{X, Y} (x, y) \,dx \,dy = 1$, so we must go ahead and take this double integral to find $k$. \\ \\
Let us change the order and first take the integral with respect to $y$. For this integral, we have two overlapping ranges due to the nature of the absolute value: $0 \leq y \leq \sqrt{x}$ and $0 \geq y \geq -\sqrt{x}$, or $-\sqrt{x} \leq y \leq \sqrt{x}$, which we can account for as follows: \\ 
$\int_{-\infty}^{\infty} f_{X, Y} (x, y) \,dy = \int_{0}^{\sqrt{x}} e^{-x} \cdot \frac{k}{\sqrt{x}} \,dy + \int_{-\sqrt{x}}^{0} e^{-x} \cdot \frac{k}{\sqrt{x}} \,dy$ \\
$= [y \cdot (e^{-x} \cdot \frac{k}{\sqrt{x}})] |_{0}^{\sqrt{x}} + [y \cdot (e^{-x} \cdot \frac{k}{\sqrt{x}})] |_{-\sqrt{x}}^{0}$ \\ 
$= (e^{-x} \cdot k) - 0 + 0 - (e^{-x} \cdot -k)$ \\ 
$= 2k \cdot e^{-x}$ \\ \\
We can now take the integral with respect to $x$:
$\int_{-\infty}^{\infty} f_{X} (x) \,dx = \int_{0}^{\infty} 2k \cdot e^{-x} \,dx$ \\ 
$ = (2k \cdot -e^{-x}) |_{0}^{\infty}$ \\ 
$ = 2k \cdot -e^{-\infty} - (2k \cdot -e^{-0})$ \\ 
$ = 0 + 2k \cdot e^{0}$ \\ 
$ = 2k = 1$ \\
From this, we can determine that \framebox[1.1\width]{\textbf{$k = \frac{1}{2}$}}

\subsection*{(b)}
We know by definition that the marginal density $f_X (x) = \int_{-\infty}^{\infty} f_{X, Y} (x, y) \,dy$ \\ 
$= \int_{-\sqrt{x}}^{\sqrt{x}} e^{-x} \cdot \frac{1}{2\sqrt{x}}$ \\ 
$= (y \cdot (e^{-x} \cdot \frac{1}{2\sqrt{x}})) |_{-\sqrt{x}}^{\sqrt{x}}$ \\ 
$= (\frac{e^{-x}}{2}) - (-\frac{e^{-x}}{2})$ \\ 
$= e^{-x}$ \\ \\
We have that \framebox[1.1\width]{\textbf{$f_x = e^{-x}$}}, which by definition allows us to identify that $X$ an \\ \framebox[1.1\width]{\textbf{exponential distribution}}.

}

\section*{4.}
{\Large 
The random vector $(X, Y)$ is said to be uniformly distributed over a region $R$ if, for some
constant $c$, its joint density is

\[
f_{X,Y}\left(x,y\right)=\begin{cases}
c& \,\mbox{if} \left(x,y\right)\in\mathbb R\\
0 & \,\mbox{otherwise,}
\end{cases}
\]

\subsection*{(a)}
We know that by definition of integral and using the bounds of $R$, \\
$\int \int_{R} \,dx \,dy = $ area of region $R$, or $\int \int_{R} 1 \,dx \,dy = $ area of region $R$. \\ 
By definition of pdf, we also know that $\int \int_{R} f_{X, Y} (x, y) \,dx \,dy = 1$. \\
Plugging in $f_{X, Y}$ in this case, we find that $\int \int_{R} c \,dx \,dy = 1$; we can use algebraic manipulation to find that \\ 
$c \cdot \int \int_{R} 1 \,dx \,dy = 1$ \\ 
and equivalently that $\int \int_{R} 1 \,dx \,dy = \frac{1}{c}$ \\
Using our first statement, we finally find the area of region $R = \frac{1}{c}$.

\subsection*{(b)}
We aim to show that $X$ and $Y$ are independent, i.e. by definition $f_X(x) \cdot f_Y(y) = f_{X, Y} (x,y)$, and uniformly distributed from $[-1, 1]$. We know that $(X, Y)$ is uniformly distributed over the square centered at (0, 0) with side length 2, which means that the values of $X$ and $Y$ will both be in the range $[-1, 1]$. We also know that since the region is a square with side length 2, the total area is $2 \cdot 2 = 4$. Using what we showed in part (a), we can now find that $4 = \frac{1}{c}$, which we can solve to find that $c = \frac{1}{4}$. We can now find $f_X(x)$ and $f_Y(y)$ respectively, and determine that they are both uniform and that our equation that tells us if $X$ and $Y$ are independent holds: \\ \\
$f_X(x) = \int_{-\infty}^{\infty} f_{X, Y} (x, y) \,dy$ \\ 
$= \int_{-1}^{1} \frac{1}{4} \,dy$ \\ 
$= (\frac{y}{4}) |_{-1}^{1}$ \\
$= \frac{1}{4} - (\frac{-1}{4})$ \\
$= \frac{1}{2}$ \\
The pdf is constant, which also tells us that $X$ is uniformly distributed on the range $[-1, 1]$. \\ \\
Equivalently, $f_Y(y) = \int_{-\infty}^{\infty} f_{X, Y} (x, y) \,dx$ \\ 
$= \int_{-1}^{1} \frac{1}{4} \,dx$ \\ 
$= (\frac{x}{4}) |_{-1}^{1}$ \\
$= \frac{1}{4} - (\frac{-1}{4})$ \\
$= \frac{1}{2}$ \\
The pdf is constant, which also tells us that $X$ is uniformly distributed on the range $[-1, 1]$. \\ \\
Finally, we can see that:
$f_(X, Y) = f_X(x) \cdot f_Y(y)$ \\
$\frac{1}{4} = \frac{1}{2} \cdot \frac{1}{2}$ \\ 
$\frac{1}{4} = \frac{1}{4}$
\\ \\
So $X$ and $Y$ are indeed independent.

\subsection*{(c)}
We aim to find the probability that $(X, Y)$ lies in the circle of radius 1 centered at the origin, i.e., $\mathrm P \,(X^2 + Y^ 2 \le 1)$. \\ \\ 
The joint pdf $f_{X, Y} = \frac{1}{4}$ as we found above. We note that the circle is completely contained within the square, so we can effectively simply take the area covered by the circle, which by definition is $\pi r^2 = \pi \cdot 1^2 = \pi$, and multiply it by the probability density function, which would give us the probability that we would end up choosing $(X, Y)$ such that it is within the circle, i.e. the probability that $(X, Y)$ lies in that wholly contained circle. Doing this, we find that this comes out to $\pi \cdot \frac{1}{4} = $ \framebox[1.1\width]{\textbf{$\frac{\pi}{4}$}}.

}

\section*{5.}
{\Large 
The time that it takes to service a car is an exponential random variable with rate 1, that is, $X \sim \text{Exp}(1)$, with $f(x) = e^{-x}$.

\subsection*{(a)}
AJ brings his car in at time 0 and MJ brings her car in at time $t$. We aim to find the probability that MJ’s car is ready before AJ’s car, that is, P$(m + t < a)$, where $a$ represents the time it takes AJ's car to be ready and $m$ represents the time it takes MJ's car to be ready. \\
$X \sim \text{Exp}(\lambda)$, where $f(x) = \lambda e^{-\lambda x} \text{ for } x \geq 0$, and 0 otherwise. \\ 
\framebox[1.1\width]{\textbf{$\frac{\pi}{4}$}}



\subsection*{(b)}
If both cars are brought in at time 0, with work starting on M. J.’s car only when A.J.’s
car has been completely serviced, what is the probability that M. J.’s car is ready before
time 2?


}

\section*{6.}
{\Large 
Suppose $X, Y$  are independent random variables with probability density functions
\[f_X (t) = f_Y (t) =\frac12e^{-|t|}, \,\,\, t \in \mathbb R\]
Find the pdf $f_Z (t)$ of $Z = X + Y$ . Hint: Consider the cases $t < 0$ and $t \ge 0$ separately. \\ \\ 
We know that for $Z = X + Y$, $f_Z(t) = \int_{-\infty}^{\infty}f_X(t - s)f_Y(x) \,ds$. \\ 
By te definition of absolute value, we know that we can have $t < 0$ and $t \geq 0$. 

}

\end{document}