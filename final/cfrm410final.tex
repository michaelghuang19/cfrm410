\documentclass{article}
\linespread{1.3}
\usepackage[margin=50pt]{geometry}
\usepackage{amsmath, amsthm, amssymb, amsthm, tikz, fancyhdr}
\pagestyle{fancy}
\renewcommand{\headrulewidth}{0pt}
\newcommand{\changefont}{\fontsize{15}{15}\selectfont}

\fancypagestyle{firstpageheader}
{
  \fancyhead[R]{\changefont Michael Huang \\ Final \\ Adekoya}
}

\begin{document}

\thispagestyle{firstpageheader}

\section*{1.}
{\Large 

\subsection*{(a)}
To find constants $a$ and $b$, we solve for expectation, that is, \\
$E(X) = \int_{-\infty}^{\infty} x \cdot f(x) \, dx $: \\ 
$\frac{6}{10} = \int_{0}^{1} t \cdot (at + bt^2) \, dt$ \\
$\frac{6}{10} = \int_{0}^{1} at^2 + bt^3 \, dt$ \\
$\frac{6}{10} = \frac{at^3}{3} + \frac{bt^4}{4} |_{0}^{1}$ \\
$\frac{6}{10} = (\frac{a \cdot 1^3}{3} + \frac{b \cdot 1^4}{4}) - 0$ \\
$\frac{6}{10} = (\frac{a}{3} + \frac{b}{4})$ \\
$36 = 20a + 15b$ \\
And to fully construct the system of equations to solve for two variables, we can also use the probability axiom that tells us that \\
$1 = \int_{-\infty}^{\infty} f(x) \, dx $: \\ 
$1 = \int_{0}^{1} at + bt^2 \, dt$ \\
$1 = \frac{at^2}{2} + \frac{bt^3}{3} |_{0}^{1}$ \\
$1 = (\frac{a \cdot 1^2}{2} + \frac{b \cdot 1^3}{3}) - 0$ \\
$1 = (\frac{a}{2} + \frac{b}{3})$ \\
$6 = 3a + 2b$ \\ \\ 
Now we have our system of equations to solve:
\[\begin{cases}
36 = 20a + 15b&\\
6 = 3a + 2b& 
\end{cases}\]
Taking the second equation, we get $b = 3 - \frac{3a}{2}$ which we can substitute back into the first equation: \\ 
$36 = 20a + 15(3 - \frac{3a}{2})$ \\
$36 = 20a + 45 - \frac{45a}{2}$ \\
$-9 = - \frac{5a}{2}$ \\
\framebox[1.1\width]{\textbf{$a = \frac{18}{5}$}} \\ 
Which in turn tells us that \\ 
$b = 3 - \frac{3 \cdot \frac{18}{5}}{2}$ \\ 
$b = 3 - \frac{3 \cdot \frac{18}{5}}{2}$ \\ 
\framebox[1.1\width]{\textbf{$b = -\frac{12}{5}$}} 

\subsection*{(b)}
By definition, $P(X \leq \frac{3}{4})$ is $\int_{-\infty}^{\frac{3}{4}} f(x) \, dx$ \\ 
$= \int_{0}^{\frac{3}{4}} \frac{18t}{5} - \frac{12t^2}{5} \, dt$ \\ 
$= \frac{9t^2}{5} - \frac{4t^3}{5} |_{0}^{\frac{3}{4}}$ \\ 
$= (\frac{9}{5} \cdot \frac{9}{16} - \frac{4}{5} \cdot \frac{27}{64}) - 0$ \\ 
$= \frac{81}{80} - \frac{108}{320}$ \\ 
= \framebox[1.1\width]{\textbf{$\frac{27}{40}$}}

\subsection*{(c)}
By definition, we know that $Var(X) = E(X^2) - E(X)^2$. We have $E(X)$, so let's find $E(X^2)$: \\ 
$E(X^2) = \int_{-\infty}^{\infty} x^2 \cdot f(x) \,dx$ \\ 
$= \int_{0}^{1} t^2 \cdot (\frac{18t}{5} - \frac{12t^2}{5}) \,dt$ \\ 
$= \int_{0}^{1} \frac{18t^3}{5} - \frac{12t^4}{5} \,dt$ \\ 
$= \frac{9t^4}{10} - \frac{12t^5}{25} |_{0}^{1}$ \\ 
$= (\frac{9 \cdot 1^4}{10} - \frac{12 \cdot 1^5}{25}) - 0$ \\ 
$= (\frac{9}{10} - \frac{12}{25})$ \\ 
$= \frac{21}{50}$ \\ \\ 
Putting this all together, we have \\ 
$Var(X) = E(X^2) - E(X)^2$ \\
$= \frac{21}{50} - (\frac{6}{10})^2$ \\
$= \frac{21}{50} - \frac{36}{100}$ \\
= \framebox[1.1\width]{\textbf{$\frac{3}{50}$}}

}

\section*{3.}
{\Large

\subsection*{(a)}
We will use the binomial distribution since we can consider the incidence of deafness in a newborn dalmatian (as cruel as it sounds) to be a success or failure, with a relatively small probability $p$ and relatively large number of trials $n$. We assume all dalmatians are newborn, and their births and hearing abilities are independent of each other. \\ \\
We aim to find $P(X \leq 70)$, given that 70 out of 200 dalmatians born were born with deafness. We use the normal distribution for the binomial, since otherwise we would require 70 terms, which gives us a distribution such that $Bin(n,p)$ is approximately $N(np, np(1-p))$, or in our case that $Bin(200, 0.3)$ is approximately $N(60, 42)$. To find $P(X \leq 70)$, we therefore simply need to find the Z-score for Z = $\frac{X - np}{\sqrt{np(1-p)}} = \frac{70-60}{\sqrt{42}} = \frac{10}{\sqrt{42}} = \sim 1.54$, which using a lookup table we can find $\Phi(1.54)$ to be \framebox[1.1\width]{\textbf{0.9382}}. This means that the puppy house should be worried about their reputation since there is a $94\%$ that the rate of deafness should be lower, which means that given a normal distribution they would be in the top $6-7\%$ in the incidence of deafness.

\subsection*{(b)}
We know that the rate of train arrival is 0.6 per hour. We can use the exponential distribution since we have a rate which we are also using to then model waiting time. We can therefore model our pdf to be $f(x) = \lambda e^{-\lambda x} = 0.6e^{-0.6x}$ with $\lambda = 0.6$, i.e. $X \sim Exp(0.6)$. \\ \\
We aim to find $Y = ln(X)$. We note that the range of $Y$ is $\{ln(x) : x \in (0, \infty)\} = (-\infty, \infty))$. We first take the cdf and find that \\
$F_Y(y) = P(Y \leq y) = P(ln(x) \leq y) = P(X \leq e^y) = F_X(e^y)$.
\\ So to find the pdf, we can differentiate this: \\
$f_Y(y) = \frac{d}{dy}F_Y(y) = \frac{d}{dy}F_X(e^y) = e^y f_x(e^y) = e^y \cdot 0.6e^{-0.6e^y} = $ \framebox[1.1\width]{\textbf{$0.6e^{y-0.6e^y}$}} for the range $-\infty < y <\infty$.
% $f(x) = 0.6e^{-0.6x}$, take the log of this: \\ 
% $ln(f(x)) = -ln(0.6)0.6x$, then perform change of variable: \\ 

\subsection*{(c)}
We know that $X$ and $Y$ can be represented as $p_X (k) = \binom{a}{k}p^k(1-p)^{a-k}$ and $p_Y (k) = \binom{b}{k}p^k(1-p)^{b-k}$, respectively. We also know that the distribution $Z = X + Y$ denotes all those combinations of $X$ and $Y$ that add up to some sum $Z$. In terms of probability, since they are independent, this means in the range of possible values for $Z$, $P(Z = i) = \sum P(X = k)P(Y = i - k)$ and vice-versa for each possible $k$ value of $X$ and $Y$ which combine to sum to $Z = i$. By this definition, we can say for the general case that \\ 
$p_Z (k) = P(Z = n) = \sum_{} \binom{a}{x} \cdot \binom{b}{y} \cdot p(Y = y) \cdot p(X = x)$ where $x + y = n$. So \\
$= \sum_{i=0}^{n} p_X(x) \cdot p_Y(y)$ where $x + y = n$ \\
$= \sum_{i=0}^{n} p_X(x) \cdot p_Y(n - x)$ \\
$= \sum_{i=0}^{n} \binom{a}{i} p^i(1-p)^{a-i} \cdot \binom{b}{k - i} p^{n-i}(1-p)^{b - (n-i)}$ \\
$= \sum_{i=0}^{n} \binom{a}{i} \binom{b}{n - i} \cdot p^i p^{n-i} \cdot (1-p)^{b - n + i} (1-p)^{a-i}$ \\
$= \sum_{i=0}^{n} \binom{a}{i} \binom{b}{n - i} \cdot p^n \cdot (1-p)^{a + b - n}$.    Here, we use the provided equation: \\
$= \binom{a + b}{n} \cdot p^n \cdot (1-p)^{a + b - n}$ \\
Which is exactly a binomial distribution $Z$ in the form $Z \sim Binom(a + b, p)$, which proves our original statement.

}

\section*{4.}
{\Large 

\subsection*{(a)}
We know that $Cov(X,Z) = \frac{Var(X + Y) - Var(X) - Var(Y)}{2}$. We know as well that $\rho_{X, Z} = \frac{1}{3} = \frac{Cov(X, Z)}{\sqrt{Var(X)Var(Z)}}$. Solving this second equation: \\
$\frac{1}{3} = \frac{Cov(X,Z)}{\sqrt{Var(X)Var(Z)}}$ \\
$\frac{1}{3} = \frac{Cov(X,Z)}{\sqrt{1 \cdot 9}}$ \\
$\frac{1}{3} = \frac{Cov(X,Z)}{3}$ \\
So we know that $Cov(X, Z) = 1$. Using this to solve for the first equation: \\
$1 = \frac{Var(X + Z) - Var(X) - Var(Z)}{2}$ \\
$1 = \frac{Var(X + Z) - 1 - 9}{2}$ \\
$2 = Var(X + Z) - 10$ \\
$Var(X + Z) = $ \framebox[1.1\width]{\textbf{12}}.

\subsection*{(b)}
We aim to find $Cov(Z, Y)$, given that $Y$ is independent from $Y + Z$. We know that $Cov(Z, Y) = \frac{Var(Z + Y) - Var(Z) - Var(Y)}{2}$. The only value we need to find is $Var(Z+Y)$. \\
We know that $Var(Z + Y + Y) = Var(Z + Y) + Var(Y) + 2Cov(Z + Y, Y)$. Since $Z+Y$ and $Y$ are independent, we know that $Var(Z + Y + Y) = Var(Z + Y - Y) = Var(Z) = 9$. Plugging this into the former equation, we find that $9 = Var(Z+Y) + 2$, or that $Var(Z + Y) = 7$ (Note that $Cov(Z+Y, Y) = 0$ since $Z+Y$ and $Y$ are independent, yet again). \\ \\
Putting this all together, we can find that $Cov(Z, Y) = \frac{7 - 9 - 2}{2} = $ \framebox[1.1\width]{\textbf{-2}}.

}

\section*{5.}
{\Large 

\subsection*{(a)}
We aim to find $E(N)$ and $Var(N)$. Let us consider the distribution of $N$ by pair $i$. For any given pair, the probability of having a person of the opposite gender paired up is $\frac{10^2}{\binom{20}{2}} = \frac{10}{19}$, since out of the $\binom{20}{2}$ ways to pick a pair, exactly $10 \cdot 10$ perfectly pair up a boy and girl. In other words, for each pair $i$, $E(N_i) = \frac{10}{19}$; implicitly this means that $N_i$ is Bernoulli such that it is 0 for not a boy-girl pair and 1 for a boy-girl pair. Therefore, to find the expected value for all 10 pairs, we would have $E(\sum_{i=1}^{10}N_i) = 10 \cdot E(N_i) =$ \framebox[1.1\width]{\textbf{$\frac{100}{19}$}}. \\ \\
Since we are considering many random variables $N_i$, we can use variance of sums to find $Var(\sum_{i=1}^{10}N_i)$ = $\sum_{i=1}^{10} Var(N_i) + \sum_{i \neq j}Cov(N_i, N_j)$. \\
To find the variance, we can find $Var(X_i) = E(X_i^2) - E(X_i)^2 = \frac{10}{19} - (\frac{10}{19})^2 = \frac{90}{361}$ \\
To find the covariance, we can take $Cov(N_i, N_j) = E(N_iN_j) - E(N_i)E(N_j)$ for $i \neq j$. We can find $E(N_iN_j) = P(N_i = \text{b-g pair})P(N_j = \text{b-g pair}) = P(N_i = \text{b-g pair}) \cdot P(N_j = \text{b-g pair} \mid N_i = \text{b-g pair}) = \frac{10^2}{\binom{20}{2}} \cdot \frac{9^2}{\binom{18}{2}} = \frac{10}{19} \cdot \frac{9}{17} = \frac{90}{323}$. We can then also find $E(N_i)E(N_j) = (\frac{10}{19})^2 = \frac{100}{361}$. We therefore have $Covar(N_i, N_j) = \frac{10}{6137}$. \\ 
Putting this all together, we find that \\
$Var(\sum_{i=1}^{10}N_i) = \sum_{i=1}^{10} Var(N_i) + \sum_{i \neq j}Cov(N_i, N_j)$ \\ 
$= 10 \cdot \frac{90}{361} + 10 \cdot 9 \cdot \frac{10}{6137}$ \\ 
$= \frac{900}{361} + \frac{900}{6137}$ \\
$= $ \framebox[1.1\width]{\textbf{$\sim$ 2.63972625061}}
 
\subsection*{(b)}
Let $G_i$ be the grade of student $i$, with each $G_i$ being an independently and identically distributed random variable each with the given pdf. We can find by definition that $F(t) = \int f(t) \, dt = \int \frac{4}{t^5} \, dt = -\frac{1}{t^4}$, but since we start at 1, we end up with $F(t) = 1 - \frac{1}{t^4}$. We can verify this graphically. \\ \\
We use the cdf definition to find $F(t) = P(G \leq t) = P(max\{G_1, G_2, G_3, G_4\} \leq t) $ \\ 
$ = P(max\{G_1, G_2, G_3, G_4\} \leq t)$ \\
$= P(G_1 \leq t, G_2 \leq t, G_3 \leq t, G_4 \leq t)$, but since all $G_i$ are iid, this equals \\
$= P(G_1 \leq t) \cdot P(G_2 \leq t) \cdot P(G_3 \leq t) \cdot P(G_4 \leq t) = (1-\frac{1}{t^4})^4$. Our cdf is therefore \framebox[1.1\width]{\textbf{$(1-\frac{1}{t^4})^4$}} in the range $1 < t < \infty$, and \framebox[1.1\width]{\textbf{0}} in the range $-\infty < t < 1$. We note that this means that at $t = \infty$, our cdf is 1. \\ \\
To find our pdf, we simply differentiate: $\frac{d}{dt} (1-\frac{1}{t^4})^4 = 4(1 - \frac{1}{t^4})^3 \cdot \frac{4}{t^5} = $  \framebox[1.1\width]{\textbf{$\frac{16(t^4-1)^3}{t^{17}}$}} for the range $1 < t < \infty$, and \framebox[1.1\width]{\textbf{0}} otherwise. \\ 

}

\end{document}